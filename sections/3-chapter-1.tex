\documentclass[../thesis.tex]{subfiles}



\begin{document}
\section{Анализ предметной области и формирование требований к информационной системе (комплексу задач)}
\subsection{Описание организации, являющейся объектом автоматизации}
\subsubsection{Экономический анализ деятельности организации}

% [h]   LaTeX decides where to put the image
% [h!]  "I want it here!"
\begin{figure}[h!]
    \centering
    \includegraphics[width=0.25\textwidth]{prue-logo}
    \caption{Логотип РЭУ \donebyin{Adobe Photoshop}}
    \label{fig:prue:logo}
\end{figure}

Логотип организации представлен на рисунке \ref{fig:prue:logo} (страница \pageref{fig:prue:logo}).

\subsubsection{Организационная структура и система управления}

...

\subsubsection{Состояние и стратегия развития информационных технологий. Состояние ИТ в организации}

...



\subsection{Анализ существующей организации бизнес- и информационных процессов}
\subsubsection{Описание существующей организации бизнес и информационных процессов}

...

\subsubsection{Анализ недостатков существующей организации бизнес- и информационных процессов}

...

\subsubsection{Формирование предложений по автоматизации бизнес-процессов}

...



\subsection{Постановка задачи автоматизации бизнес-процессов}
\subsubsection{Цели и задачи проекта автоматизации бизнес-процессов. Сущность комплекса задач, место проектируемого комплекса задач в информационной системе}

...

\subsubsection{Построение и обоснование модели новой организации бизнес-процессов}

...

\subsubsection{Спецификация функциональных требований к информационной системе}

...

\subsubsection{Спецификация и обоснование нефункциональных требований}

...



\subsection{Календарно-ресурсное планирование проекта, анализ бюджетных ограничений и рисков}

...

\end{document}